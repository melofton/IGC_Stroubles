\documentclass[]{article}
\usepackage{lmodern}
\usepackage{amssymb,amsmath}
\usepackage{ifxetex,ifluatex}
\usepackage{fixltx2e} % provides \textsubscript
\ifnum 0\ifxetex 1\fi\ifluatex 1\fi=0 % if pdftex
  \usepackage[T1]{fontenc}
  \usepackage[utf8]{inputenc}
\else % if luatex or xelatex
  \ifxetex
    \usepackage{mathspec}
  \else
    \usepackage{fontspec}
  \fi
  \defaultfontfeatures{Ligatures=TeX,Scale=MatchLowercase}
\fi
% use upquote if available, for straight quotes in verbatim environments
\IfFileExists{upquote.sty}{\usepackage{upquote}}{}
% use microtype if available
\IfFileExists{microtype.sty}{%
\usepackage{microtype}
\UseMicrotypeSet[protrusion]{basicmath} % disable protrusion for tt fonts
}{}
\usepackage[margin=1in]{geometry}
\usepackage{hyperref}
\hypersetup{unicode=true,
            pdftitle={`New' Conductivity HOBO monthly plots},
            pdfauthor={Mary Lofton \& Lauren Wind},
            pdfborder={0 0 0},
            breaklinks=true}
\urlstyle{same}  % don't use monospace font for urls
\usepackage{color}
\usepackage{fancyvrb}
\newcommand{\VerbBar}{|}
\newcommand{\VERB}{\Verb[commandchars=\\\{\}]}
\DefineVerbatimEnvironment{Highlighting}{Verbatim}{commandchars=\\\{\}}
% Add ',fontsize=\small' for more characters per line
\usepackage{framed}
\definecolor{shadecolor}{RGB}{248,248,248}
\newenvironment{Shaded}{\begin{snugshade}}{\end{snugshade}}
\newcommand{\AlertTok}[1]{\textcolor[rgb]{0.94,0.16,0.16}{#1}}
\newcommand{\AnnotationTok}[1]{\textcolor[rgb]{0.56,0.35,0.01}{\textbf{\textit{#1}}}}
\newcommand{\AttributeTok}[1]{\textcolor[rgb]{0.77,0.63,0.00}{#1}}
\newcommand{\BaseNTok}[1]{\textcolor[rgb]{0.00,0.00,0.81}{#1}}
\newcommand{\BuiltInTok}[1]{#1}
\newcommand{\CharTok}[1]{\textcolor[rgb]{0.31,0.60,0.02}{#1}}
\newcommand{\CommentTok}[1]{\textcolor[rgb]{0.56,0.35,0.01}{\textit{#1}}}
\newcommand{\CommentVarTok}[1]{\textcolor[rgb]{0.56,0.35,0.01}{\textbf{\textit{#1}}}}
\newcommand{\ConstantTok}[1]{\textcolor[rgb]{0.00,0.00,0.00}{#1}}
\newcommand{\ControlFlowTok}[1]{\textcolor[rgb]{0.13,0.29,0.53}{\textbf{#1}}}
\newcommand{\DataTypeTok}[1]{\textcolor[rgb]{0.13,0.29,0.53}{#1}}
\newcommand{\DecValTok}[1]{\textcolor[rgb]{0.00,0.00,0.81}{#1}}
\newcommand{\DocumentationTok}[1]{\textcolor[rgb]{0.56,0.35,0.01}{\textbf{\textit{#1}}}}
\newcommand{\ErrorTok}[1]{\textcolor[rgb]{0.64,0.00,0.00}{\textbf{#1}}}
\newcommand{\ExtensionTok}[1]{#1}
\newcommand{\FloatTok}[1]{\textcolor[rgb]{0.00,0.00,0.81}{#1}}
\newcommand{\FunctionTok}[1]{\textcolor[rgb]{0.00,0.00,0.00}{#1}}
\newcommand{\ImportTok}[1]{#1}
\newcommand{\InformationTok}[1]{\textcolor[rgb]{0.56,0.35,0.01}{\textbf{\textit{#1}}}}
\newcommand{\KeywordTok}[1]{\textcolor[rgb]{0.13,0.29,0.53}{\textbf{#1}}}
\newcommand{\NormalTok}[1]{#1}
\newcommand{\OperatorTok}[1]{\textcolor[rgb]{0.81,0.36,0.00}{\textbf{#1}}}
\newcommand{\OtherTok}[1]{\textcolor[rgb]{0.56,0.35,0.01}{#1}}
\newcommand{\PreprocessorTok}[1]{\textcolor[rgb]{0.56,0.35,0.01}{\textit{#1}}}
\newcommand{\RegionMarkerTok}[1]{#1}
\newcommand{\SpecialCharTok}[1]{\textcolor[rgb]{0.00,0.00,0.00}{#1}}
\newcommand{\SpecialStringTok}[1]{\textcolor[rgb]{0.31,0.60,0.02}{#1}}
\newcommand{\StringTok}[1]{\textcolor[rgb]{0.31,0.60,0.02}{#1}}
\newcommand{\VariableTok}[1]{\textcolor[rgb]{0.00,0.00,0.00}{#1}}
\newcommand{\VerbatimStringTok}[1]{\textcolor[rgb]{0.31,0.60,0.02}{#1}}
\newcommand{\WarningTok}[1]{\textcolor[rgb]{0.56,0.35,0.01}{\textbf{\textit{#1}}}}
\usepackage{graphicx,grffile}
\makeatletter
\def\maxwidth{\ifdim\Gin@nat@width>\linewidth\linewidth\else\Gin@nat@width\fi}
\def\maxheight{\ifdim\Gin@nat@height>\textheight\textheight\else\Gin@nat@height\fi}
\makeatother
% Scale images if necessary, so that they will not overflow the page
% margins by default, and it is still possible to overwrite the defaults
% using explicit options in \includegraphics[width, height, ...]{}
\setkeys{Gin}{width=\maxwidth,height=\maxheight,keepaspectratio}
\IfFileExists{parskip.sty}{%
\usepackage{parskip}
}{% else
\setlength{\parindent}{0pt}
\setlength{\parskip}{6pt plus 2pt minus 1pt}
}
\setlength{\emergencystretch}{3em}  % prevent overfull lines
\providecommand{\tightlist}{%
  \setlength{\itemsep}{0pt}\setlength{\parskip}{0pt}}
\setcounter{secnumdepth}{0}
% Redefines (sub)paragraphs to behave more like sections
\ifx\paragraph\undefined\else
\let\oldparagraph\paragraph
\renewcommand{\paragraph}[1]{\oldparagraph{#1}\mbox{}}
\fi
\ifx\subparagraph\undefined\else
\let\oldsubparagraph\subparagraph
\renewcommand{\subparagraph}[1]{\oldsubparagraph{#1}\mbox{}}
\fi

%%% Use protect on footnotes to avoid problems with footnotes in titles
\let\rmarkdownfootnote\footnote%
\def\footnote{\protect\rmarkdownfootnote}

%%% Change title format to be more compact
\usepackage{titling}

% Create subtitle command for use in maketitle
\providecommand{\subtitle}[1]{
  \posttitle{
    \begin{center}\large#1\end{center}
    }
}

\setlength{\droptitle}{-2em}

  \title{`New' Conductivity HOBO monthly plots}
    \pretitle{\vspace{\droptitle}\centering\huge}
  \posttitle{\par}
    \author{Mary Lofton \& Lauren Wind}
    \preauthor{\centering\large\emph}
  \postauthor{\par}
      \predate{\centering\large\emph}
  \postdate{\par}
    \date{Mar.~22, 2020}


\begin{document}
\maketitle

\#load packages

\begin{Shaded}
\begin{Highlighting}[]
\CommentTok{#install.packages("pacman")}
\NormalTok{pacman}\OperatorTok{::}\KeywordTok{p_load}\NormalTok{(tidyverse, lubridate, zoo,cowplot)}
\end{Highlighting}
\end{Shaded}

\#read in Bridge 1 data conductivity and discharge data

\begin{Shaded}
\begin{Highlighting}[]
\NormalTok{B1 <-}\StringTok{ }\KeywordTok{read_csv}\NormalTok{(}\StringTok{"./B1_QAQC_3_20_2020_LLW.csv"}\NormalTok{) }\OperatorTok
\StringTok{  }\KeywordTok{mutate}\NormalTok{(}\DataTypeTok{DateTime =} \KeywordTok{as.POSIXct}\NormalTok{(DateTime,}\DataTypeTok{format=}\StringTok{"%m/%d/%Y %H:%M"}\NormalTok{,}\DataTypeTok{tz=}\KeywordTok{Sys.timezone}\NormalTok{()))}
\end{Highlighting}
\end{Shaded}

\begin{verbatim}
## Parsed with column specification:
## cols(
##   DateTime = col_character(),
##   Full.Conductivity = col_double(),
##   Temp_F = col_double(),
##   Sp.Conductivity = col_double(),
##   Lvl_m = col_double(),
##   Discharge = col_double(),
##   Flag_Lvl_m = col_double(),
##   Flag_Temp_F = col_double(),
##   Flag_SpCond_uScm = col_double()
## )
\end{verbatim}

\#Visualize how much data this QA is flagging

\begin{Shaded}
\begin{Highlighting}[]
\CommentTok{##HISTOGRAM OF SPCOND}
\KeywordTok{hist}\NormalTok{(B1}\OperatorTok{$}\NormalTok{Sp.Conductivity)}
\end{Highlighting}
\end{Shaded}

\includegraphics{Make_pngs_Cond_HOBO_files/figure-latex/unnamed-chunk-3-1.pdf}

\begin{Shaded}
\begin{Highlighting}[]
\CommentTok{##HISTOGRAM OF LVL_M}
\KeywordTok{hist}\NormalTok{(B1}\OperatorTok{$}\NormalTok{Lvl_m,  }\DataTypeTok{breaks =} \KeywordTok{seq}\NormalTok{(}\DecValTok{0}\NormalTok{,}\DecValTok{2}\NormalTok{,}\DataTypeTok{by =} \FloatTok{0.01}\NormalTok{), }\DataTypeTok{main =} \StringTok{"Red line is if we cut off at 0.02 m"}\NormalTok{)}
\KeywordTok{abline}\NormalTok{(}\DataTypeTok{v =} \FloatTok{0.02}\NormalTok{, }\DataTypeTok{lwd =} \DecValTok{2}\NormalTok{, }\DataTypeTok{col =} \StringTok{"red"}\NormalTok{)}
\end{Highlighting}
\end{Shaded}

\includegraphics{Make_pngs_Cond_HOBO_files/figure-latex/unnamed-chunk-3-2.pdf}

\begin{Shaded}
\begin{Highlighting}[]
\CommentTok{#on manual inspection, these low-flow values are from Aug-Oct 2019}
\CommentTok{# check <- B1 %>%}
\CommentTok{#   filter(Lvl_m <=0.1)}

\CommentTok{#check relationship btwn. discharge and cond.}
\KeywordTok{plot}\NormalTok{(B1}\OperatorTok{$}\NormalTok{Discharge,B1}\OperatorTok{$}\NormalTok{Sp.Conductivity)}
\end{Highlighting}
\end{Shaded}

\includegraphics{Make_pngs_Cond_HOBO_files/figure-latex/unnamed-chunk-3-3.pdf}

\begin{Shaded}
\begin{Highlighting}[]
\CommentTok{#now on log scale}
\KeywordTok{plot}\NormalTok{(}\KeywordTok{log}\NormalTok{(B1}\OperatorTok{$}\NormalTok{Discharge),}\KeywordTok{log}\NormalTok{(B1}\OperatorTok{$}\NormalTok{Sp.Conductivity))}
\end{Highlighting}
\end{Shaded}

\includegraphics{Make_pngs_Cond_HOBO_files/figure-latex/unnamed-chunk-3-4.pdf}

\#Pngs with a Lvl\_m cutoff at 0.02 m

\begin{Shaded}
\begin{Highlighting}[]
\CommentTok{##DATA VIZ FOR DISCHARGE AND CONDUCTIVITY ONLY}
\NormalTok{B1_viz <-}\StringTok{ }\NormalTok{B1 }\OperatorTok
\StringTok{  }\KeywordTok{mutate}\NormalTok{(}\DataTypeTok{Discharge_cms_perc =} \KeywordTok{ifelse}\NormalTok{(Discharge }\OperatorTok{>=}\StringTok{ }\KeywordTok{quantile}\NormalTok{(Discharge, }\DataTypeTok{probs =} \FloatTok{0.90}\NormalTok{, }\DataTypeTok{na.rm =} \OtherTok{TRUE}\NormalTok{),}\DecValTok{1}\NormalTok{,}\DecValTok{0}\NormalTok{),}
         \DataTypeTok{Flag_SpCond_uScm =} \KeywordTok{ifelse}\NormalTok{(}\OperatorTok{!}\KeywordTok{is.na}\NormalTok{(Lvl_m) }\OperatorTok{&}\StringTok{ }\NormalTok{Lvl_m }\OperatorTok{<}\StringTok{ }\FloatTok{0.02}\NormalTok{,}\DecValTok{1}\NormalTok{,Flag_SpCond_uScm)) }\OperatorTok
\StringTok{  }\KeywordTok{select}\NormalTok{(DateTime, Sp.Conductivity, Flag_SpCond_uScm, Discharge, Discharge_cms_perc) }\OperatorTok
\StringTok{  }\KeywordTok{mutate}\NormalTok{(}\DataTypeTok{Month =} \KeywordTok{month}\NormalTok{(DateTime),}
         \DataTypeTok{Year =} \KeywordTok{year}\NormalTok{(DateTime),}
         \DataTypeTok{Flag_SpCond_uScm =} \KeywordTok{as.factor}\NormalTok{(Flag_SpCond_uScm),}
         \DataTypeTok{Discharge_cms_perc =} \KeywordTok{as.factor}\NormalTok{(Discharge_cms_perc)) }

\NormalTok{months <-}\StringTok{ }\KeywordTok{c}\NormalTok{(}\DecValTok{6}\OperatorTok{:}\DecValTok{12}\NormalTok{,}\DecValTok{1}\OperatorTok{:}\DecValTok{3}\NormalTok{)}
  
  \ControlFlowTok{for}\NormalTok{ (j }\ControlFlowTok{in} \DecValTok{1}\OperatorTok{:}\KeywordTok{length}\NormalTok{(months))\{}
    
\NormalTok{    month <-}\StringTok{ }\KeywordTok{subset}\NormalTok{(B1_viz, Month }\OperatorTok{==}\StringTok{ }\NormalTok{months[j])}
    
\NormalTok{    p1 <-}\StringTok{ }\KeywordTok{ggplot}\NormalTok{(}\DataTypeTok{data =}\NormalTok{ month, }\KeywordTok{aes}\NormalTok{(}\DataTypeTok{x =}\NormalTok{ DateTime,}\DataTypeTok{y =}\NormalTok{ Sp.Conductivity, }\DataTypeTok{color =}\NormalTok{ Flag_SpCond_uScm))}\OperatorTok{+}
\StringTok{      }\KeywordTok{geom_point}\NormalTok{()}\OperatorTok{+}
\StringTok{      }\KeywordTok{theme_bw}\NormalTok{() }\OperatorTok{+}
\StringTok{      }\KeywordTok{ggtitle}\NormalTok{(}\KeywordTok{paste}\NormalTok{(months[j],}\StringTok{"  1=out of water"}\NormalTok{,}\DataTypeTok{sep =} \StringTok{""}\NormalTok{))}
    
\NormalTok{    p2 <-}\StringTok{ }\KeywordTok{ggplot}\NormalTok{(}\DataTypeTok{data =}\NormalTok{ month, }\KeywordTok{aes}\NormalTok{(}\DataTypeTok{x =}\NormalTok{ DateTime,}\DataTypeTok{y =}\NormalTok{ Discharge, }\DataTypeTok{color =}\NormalTok{ Discharge_cms_perc))}\OperatorTok{+}
\StringTok{      }\KeywordTok{geom_point}\NormalTok{()}\OperatorTok{+}
\StringTok{      }\KeywordTok{theme_bw}\NormalTok{() }\OperatorTok{+}
\StringTok{      }\KeywordTok{ggtitle}\NormalTok{(}\KeywordTok{paste}\NormalTok{(months[j],}\StringTok{"  1=90th perc."}\NormalTok{,}\DataTypeTok{sep =} \StringTok{""}\NormalTok{))}
    
\NormalTok{    p3 <-}\StringTok{ }\KeywordTok{plot_grid}\NormalTok{(p1,p2,}\DataTypeTok{align =} \StringTok{"hv"}\NormalTok{, }\DataTypeTok{nrow =} \DecValTok{2}\NormalTok{, }\DataTypeTok{ncol =} \DecValTok{1}\NormalTok{)}
    
    \KeywordTok{print}\NormalTok{(p3)}
    \CommentTok{# ggsave(p3, filename = paste0("C:/Users/Mary Lofton/Documents/IGC/Stroubles_project/EDI_data_viz/Cond+Discharge/",paste("Cond_Discharge",yrz[i],months[j],sep = "-"),".png"),height = 14, width = 14, units = "in", scale = 0.5)}
\NormalTok{  \}}
\end{Highlighting}
\end{Shaded}

\includegraphics{Make_pngs_Cond_HOBO_files/figure-latex/unnamed-chunk-4-1.pdf}
\includegraphics{Make_pngs_Cond_HOBO_files/figure-latex/unnamed-chunk-4-2.pdf}
\includegraphics{Make_pngs_Cond_HOBO_files/figure-latex/unnamed-chunk-4-3.pdf}
\includegraphics{Make_pngs_Cond_HOBO_files/figure-latex/unnamed-chunk-4-4.pdf}
\includegraphics{Make_pngs_Cond_HOBO_files/figure-latex/unnamed-chunk-4-5.pdf}
\includegraphics{Make_pngs_Cond_HOBO_files/figure-latex/unnamed-chunk-4-6.pdf}
\includegraphics{Make_pngs_Cond_HOBO_files/figure-latex/unnamed-chunk-4-7.pdf}
\includegraphics{Make_pngs_Cond_HOBO_files/figure-latex/unnamed-chunk-4-8.pdf}
\includegraphics{Make_pngs_Cond_HOBO_files/figure-latex/unnamed-chunk-4-9.pdf}
\includegraphics{Make_pngs_Cond_HOBO_files/figure-latex/unnamed-chunk-4-10.pdf}


\end{document}
